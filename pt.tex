\documentclass{cv}
\pagenumbering{gobble}

% My info
\title{Curriculum Vitae}
\author{Felipe Rodopoulos de Oliveira}
\phone{+55 (61) 99234 2360}
\location{Brasília, Brazil}
\githubuser{rodopoulos}
\email{felipekss@gmail.com}

\begin{document}
  \maketitle
  \vspace{-.75cm}
  \section{Formação}
  \begin{chronoitem}
    {Mestrando em Informática} {Universidade de Brasília} {2017 - Atual}
    Tema: formalização e verificação de protocolos de segurança e provadores de teorema
  \end{chronoitem}

  \begin{chronoitem}
    {Bacharelado em Ciência da Computação} {Universidade de Brasília} {2010 - 2015}
    Projeto de Graduação: WebSpy - uma ferramenta para monitamento e visualização de tráfego web de um alvo em tempo real \\
    \textbf{Premiado como Aluno Destaque do semestre}
  \end{chronoitem}

  \begin{chronoitem}
    {Adicional}{}{}
    \begin{itemize}\setlength\itemsep{.5em}
      \item[] Raspberry Pi Básico \hfill (16 horas - 2016)
      \item[] Front-end Programming with Javascript and jQuery \hfill (40 horas - 2015)
      \item[] Java and Object Oriented Programming \hfill (40 horas - 2013)
    \end{itemize}
  \end{chronoitem}
  \vspace{-.5cm}

  \section{Experiência}
  \begin{chronoitem}
    {Administrador de Rede}{Agência Espacial Brasileira}{2013-2016}
    Infraestrutura de DNS e servidores de aplicação com alta disponibilidade, gestão de segurança, firewalls, switches e redes wireless, administração de Windows AD, deploy de aplicações corporativas (comerciais e open-source), monitoramento de rede (Zabbix e Nagios), administração de banco de dados e auditoria de rede. Desenvolvimento web usando WordPress e Javascript.
  \end{chronoitem}

  \begin{chronoitem}
    {Coordenador de Projetos}{CJR (Empresa Júnior de Computação)}{2011-2013}
    Desenvolvimento web (Joomla, Codeigniter, HTML/CSS, MySQL, JavaScript e jQuery), desenvolvimento e aplicação de cursos (Moodle, PHP, jQuery, MySQL e SEO) e gerência de equipe e projetos.  
    \textbf{Premiado como melhor desenvolvedor na gestão 2012/2013}
  \end{chronoitem}
  \vspace{-.5cm}

  \section{Habilidades}
  \begin{itemize}
    \item[] \textbf{Linguagens:} C/C++, Go, Python, PHP, Javascript e PHP
    \item[] \textbf{Ferramentas:} HTML, CSS (Sass), Docker, NPM, Gulp, cmake e Git
    \item[] \textbf{Frameworks:} Flask, jQuery, Codeigniter
    \item[] \textbf{Banco de dados:} MySQL, PostgreSQL, MongoDB
    \item[] \textbf{Software:} \LaTeX, Wordpress, Unix, Apache, Nginx, BIND9, dnsmasq, Joomla, Moodle, Windows AD, OpenLDAP, Nagios, Samba
  \end{itemize}

  \section{Projetos}
  \begin{chronoitem}{the\_theme}{projeto pessoal}{}
    Um template de Wordpress, baseado em HTML, Sass e Gulp, feito para otimizar e acelerar meu trabalho com freelances
  \end{chronoitem}

  \begin{chronoitem}{webspy}{graduation final project}{}
    Ferramenta em C++, utilizada para redirecionar e reproduzir uma sessão HTTP ou HTTPS de uma dada vítima em uma rede local
  \end{chronoitem}

  \begin{chronoitem}{compilador cnet}{Trabalho final de Compiladores}{}
    Compilador em Flex, Bison e C para a linguagem hipotética C-net, extensão de parte da linguagem C, que entende endereços IPv4, IPv6 e MAC, gerando um código de três endereços
  \end{chronoitem}

  \begin{chronoitem}{fd8-torrent}{Trabalho final de Teleinformática e Redes}{}
    Sistema de distribuição e compartilhamento de arquivos por um servidor HTTP feito em C++ e SDL e uma interface web em HTML, CSS e Javascript
  \end{chronoitem}

\end{document}
